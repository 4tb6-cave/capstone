% Meeting 6
	\pagebreak
	\section*{Oct. 16, 2025}
	\addcontentsline{toc}{section}{Meeting 6}

	\textbf{Attendees:} Abdul, Tabish, Berk, Andrew \\
	\textbf{Notetaker:} Nicholas

\subsection*{Team Reading Week Sync Up}
	\subsubsection*{Agenda}
		\begin{itemize}
			\item Figure out what we need to do for the electives over the next two weeks.
			\item Catch up on who has done what
		\end{itemize}

	\subsubsection*{Minutes}
		\begin{itemize}
			\item PoC and contract is main deliverables
			\item Example contracts have been posted, they are short
			\item Some are more formal, could copy and paste format from examples
			\item Middle ground, or see what it looks like
			\item Currently using sensors with companies algorithm, only shows realtime map, no persistence
			\item Apparently are some sensors from the same company that would work well but are out of stock
			\item Electrical - Unsure what boards 2/3 looks like, may just be in bin with other stuff. Not sure if theyre still needed
			\item Figure out purposes of all the boards, write down
			\item Old groups code has also been added, need to go over their code to understand sensor interfaces
			\item Invited Dr. Giamou to the repo
			\item Created a github milestone to organize PoC
			\item Need fusion from IMU to create persistent map from ToF- should be main goal of PoC
			\item Ask Andrew to bring built PCBs for Nicholas
			\item IMU is BNO055
			\item Leaning towards using Pi4, reduced power requirements and we don't need the power onboard
			\item Some licensing issues for Sonarqube with C++?
			\item Cython again an option, or alternative static analysis tools
			\item Textbooks- will add a page with links/resources for us to use - \url{http://asrl.utias.utoronto.ca/~tdb/bib/barfoot_ser24.pdf}
			\item Go over SLAM, sensor fusion 
			\item Document sensor setup so others can replicate- try docker image
		\end{itemize}

	\subsubsection*{Action Items}
		\begin{itemize}
			\item Proof of concept due in 2 weeks
			\item Draft contract as a part of that
			\item Abdul will work on sensors - try and get raw data from sensor, analyze somehow into a single frame/capture
			\item Nicholas will dissect schematics
			\item Everyone should get familiar with last years' code, algorithms
		\end{itemize}


