% Meeting 5
	\pagebreak
	\section*{Oct. 9, 2025}
	\addcontentsline{toc}{section}{TA Meeting 1, Meeting 5 \textit{(with Dr. Giamou)}}

	\textbf{C.A.V.E Attendees:} Abdul, Tabish, Berk, Andrew, Nicholas\\
	\textbf{Notetaker:} Nicholas

	\subsection*{TA Meeting - 3:45pm - Dr. Wassyng, Vasily Kapustin}
		\subsubsection*{Agenda}
			\begin{itemize}
				\item Verify PoC deliverables
				\item Discuss main branch protection
				\item Talk about the previous years' attempt, what we can take from that
			\end{itemize}

		\subsubsection*{Minutes}
			\begin{itemize}
				\item Went over summary of the project with Dr. Wassyng and Dr. Rapoport, cave mapping goal
				\item Got a hold of one of last years' group. On the topic of reuse, it's up to us how much to reuse. Build on other projects with most projects, just acknowledge.
				\item Last year didnt finish algos, apparently had trouble with the helmet. It didn't see everything they wanted, was unnatural to use.
				\item Talk with Giamou a lot to start- this will become difficult
				\item Easier to write your own algorithms than trying to debug someone else's. Will be easier to start fresh
				\item ToF sensor works with vendor software, in Thode
				\item Proof of concept deliverables- not a prototype, just think about what is critical for success. Show some kind of proof that we can overcome the main hurdle
				\item Quick and dirty, feasibility proof. Does not have to be used in final protoypes/iterations. 
				\item Informal presentation, tell problem (1-2 of them) show why it works. Previous videos to base off off
				\item Written contract: submitted Friday Oct 31. Things that we are guaranteeing that we will have done. High level goals, no nice-to-haves, underpromise, overdeliver.
				\item \textit{Berk arrives}
				\item Written contract is conditions for success; shouldnt be trivial.
				\item Do a good job on version control. Try and get (code) architecture done early, set interfaces (internal sw, sw/hw).
				\item Do sfwr early- some groups skimp on unit testing and just do integration. Having the lower level verification eases debugging
				\item Backups are a good idea, disable force pushes (history rewriting).
				\item Chat with Vasily, he's very good with software (and toasters). He is not usually at these meetings, more of an advisor. (they have other classes and commitments)
				\item Have a reasonably regular meeting with him. He'll be vocal about changes we should make.
				\item \textit{Reviewing} and reusing their electronics counts for the electrical requirement- "debugging is more difficult than writing"
				\item Dont be as smart as possible when coding, because then you're not smart enough to debug it. Keep it simple. Optimize after it works.
			\end{itemize}

		\subsubsection*{Action Items}
			\begin{itemize}
				\item Work on proof of concept over reading week break (due week of Oct 27)
				\item Work on written demo contract (due Oct 31)
				\item Talk with Dr. Giamou and our TA Vasily
			\end{itemize}

	\subsection*{Supervisor Meeting - 4:10-5:00pm}
		\subsubsection*{Agenda}
			\begin{itemize}
				\item Plan future meetings
				\item Identify points of clarification
				\item Talk with Dr. Giamou about what we should do for the PoC
			\end{itemize}

		\subsubsection*{Minutes}
			\begin{itemize}
				\item Before meeting with Giamou, talking about reading week plans
				\item Will meet online during the regular slot on reading week, to check in
				\item Simple slideshow with some of last years' stuff, whatever work we have done for the proof of concept.
				\item electrical distribution board appears to have been open sourced
				\item PoC: show sensor working with a laptop or Pi, fusion?
				\item Stick with MIT license
				\item Review the CI/CD settings so that we don't get 3 emails every time documentation is changed
				\item Should create a milestone for the PoC and start putting issues under that to distribute work
				\item Try out decomposing issues, see how the hierarchy works in Github- start big, break into small
				\item New/different sensors: keep current for PoC, change later if needed to achieve goals or requirements
				\item Make issues as we think of them, assign them as needed
				\item Can reconfigure sonarqube to ignore documents file, may disable actions for now
				\item Tabish can also help with unit tests; more of a later problem. Will know more about what to test once we have more code completed.
				\item \textit{Meeting with Dr Giamou}
				\item Updating, telling him about deliverables. Need to figure out what the PoC pain points are.
				\item Getting a couple of the sensors integrated, have 3 weeks to work on PoC
				\item What went well last year and what didnt go well?
				\item Werent able to do point cload alignment with ToF sensors. Try and get them up and running or decide on different sensor, or choose brighter env.
				\item We found that ToF works, could work with the vendors software. Need to get it clean enough to do ICP, fuse with IMU.
				\item IMU as complimentary filter, use only for orientation. Gyro/yaw
				\item 2D lidar was working well indoor. ToF is supposed to help with low light
				\item To make point cloud, imu gives orientation, accumulate points from one position. Need ToF to do full 6dof.
				\item For us, lidar gets larger FoV point cloud that isnt ToF, which have narrower views.
				\item Start with ToF, get them up and running, ICP with them. DOn't need to fuse IMU yet, use with 2d lidar to build point cloud.
				\item Pose: orientation+position- 6 axes
				\item Andrew has academic resources for this form last year, textbook on state estimation. Andrew's literature review on subterranean mapping.
				\item *Andrew*- send us the resources
				\item Open loop estimation, trusting odometry from ICP.
				\item Did a demo with the ToF sensor, looks pretty good. Does well in middling lighting within Giamou's office.
				\item Consider other battery chemistries- LiFePO4?
				\item Decide on coverage of the two sensors, what fields of view are important. Side to side? Forward? experiment. No need for stereo, overlapping could interfere though
				\item Consider button to control when imaging, no need for constant movement
				\item Existing algos to fuse data, openCV, ICP
				\item For combining IMU with point cloud, do extended filtering. No loop closures. 2d slice not for space, for mapping
				\item Next meeting should be right after reading week or when PoC is mostly done, 2-3wks from now, would have to be online once his daughter is born. Meet online in November.
				\item Giamou is in person M/T/Th
				\item Mondays are a good day- 1:30-3:30, or 10:30-12:30. 20th or 27th depending on PoC, likely at 2:30pm.
			\end{itemize}

		\subsubsection*{Action Items}
			\begin{itemize}
				\item PoC due Oct 30 - see above, mainly work on ToF. IMU later, then 2d lidar.
				\item Add Dr. Giamou to the GitHub - username mattgiamou
			\end{itemize}


