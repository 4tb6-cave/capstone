\documentclass[12pt]{article}

\usepackage[round]{natbib}
\usepackage{hyperref}
\hypersetup{colorlinks=true,
    linkcolor=blue,
    citecolor=blue,
    filecolor=blue,
    urlcolor=blue,
    unicode=false}
\urlstyle{same}

\usepackage{enumitem,amssymb}
\newlist{todolist}{itemize}{2}
\setlist[todolist]{label=$\square$}
\usepackage{pifont}
\newcommand{\cmark}{\ding{51}}%
\newcommand{\xmark}{\ding{55}}%
\newcommand{\done}{\rlap{$\square$}{\raisebox{2pt}{\large\hspace{1pt}\cmark}}%
\hspace{-2.5pt}}
\newcommand{\wontfix}{\rlap{$\square$}{\large\hspace{1pt}\xmark}}

\begin{document}

\title{POC Checklist}
\author{Spencer Smith}
\date{\today}

\maketitle

% Show an item is done by   \item[\done] Frame the problem
% Show an item will not be fixed by   \item[\wontfix] profit

The Proof of Concept Demonstration is an important milestone for assessing the
state of the project, both with respect to the project scope and the functioning
of the team itself.

\begin{itemize}

\item Project Status/scope
  \begin{todolist}
  \item Team has a name (If the name is still TBD, create a name and issue a
  pull request in the course repo).
  \item Risks for POC demo are up to date in Problem Statement document (any
  feedback from the TA has been addressed).
  \item Planned POC demo is up to date in Problem Statement document (any
  feedback from the TA has been addressed).
  \item Have a clear plan on which team members will show which parts of the POC demo.
  \item Know which devices or devices will be used for the demo.
  \item The team has practiced the demo.
  \item Demo will include a live demonstration of running code.
  \item If the planned demo will be unsuccessful, the team has some ideas on how
  to fix the problem, or redefine the scope of the project.
  \end{todolist}

\item Team Status
  \begin{todolist}
    
  \item GitHub contributions page shows all team members with similar number of
  commits, lines added, lines deleted.  When there is a difference, you should
  determine if there is a team cooperation problem, or simply a GitHub book
  keeping explanation.
  \item If your team has been coauthoring work, but not using co-author commits
  in GitHub, you plan on doing so going forward.
  \item Team has thought about how work will be divided between team members
  going forward.  Revisit development plan if there are any changes.
  \end{todolist}

\item Issue Tracker
  \begin{todolist}
    
  \item Some of the issues created by other teams for the previous deliverables
  have been closed.
  \item All closed issues include an explanation of why the issue was closed.
  The explanation could be the issue was closed because of an accepted pull
  request, or traceability to the specific commit hash that closed the issue.
  \item Issues aren't too big.  They are doable by a single person with a
  reasonable amount of time.  What needs to happen for the issues to be closed
  is clear.
  \item There are examples of issues created by the team have been closed.
  \item Issues created for team meetings (labelled so easy to find).
  \item Issues created for lectures (labelled so easy to find).
  \item Issues created for meetings with supervisor (labelled so easy to find).
  \end{todolist}

\item CI/CD
  \begin{todolist}
    
  \item Planning for CI/CD has started.
  
  \end{todolist}

\item Repo
  \begin{todolist}
  \item README file on landing page has content
  \item Branches are used as appropriate
  \end{todolist}

\item Demonstration
\begin{todolist}
  \item Team has practice demo
  \item Team does NOT have slides
  \item Team knows who will be speaking
  \item Team knows which computer they will be using
  \end{todolist}
\end{itemize}

\end{document}
