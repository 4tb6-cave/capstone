\documentclass{article}

\usepackage{tabularx}
\usepackage{booktabs}

\title{Problem Statement and Goals\\\progname}

\author{\authname}

\date{}

\input{../Comments}
%% Common Parts

\newcommand{\progname}{Mechatronics} % PUT YOUR PROGRAM NAME HERE
\newcommand{\authname}{Team 8, C.A.V.E
\\ Abdul Rahim Khan
\\ Andrew Brink
\\ Gokberk Yilmaz
\\ Nicholas Trimble
\\ Tabish Faisal} % AUTHOR NAMES                  

\usepackage{hyperref}
    \hypersetup{colorlinks=true, linkcolor=blue, citecolor=blue, filecolor=blue,
                urlcolor=blue, unicode=false}
    \urlstyle{same}
                                


\begin{document}

\maketitle

\begin{table}[hp]
\caption{Revision History} \label{TblRevisionHistory}
\begin{tabularx}{\textwidth}{llX}
\toprule
\textbf{Date} & \textbf{Developer(s)} & \textbf{Change}\\
\midrule
Date1 & Name(s) & Description of changes\\
Date2 & Name(s) & Description of changes\\
... & ... & ...\\
\bottomrule
\end{tabularx}
\end{table}

\section{Problem Statement}

\wss{You should check your problem statement with the
\href{https://github.com/smiths/capTemplate/blob/main/docs/Checklists/ProbState-Checklist.pdf}
{problem statement checklist}.} 

\wss{You can change the section headings, as long as you include the required
information.}

\subsection{Problem}

\subsection{Inputs and Outputs}

\wss{Characterize the problem in terms of ``high level'' inputs and outputs.  
Use abstraction so that you can avoid details.}

\subsection{Stakeholders}

\subsection{Environment}

\wss{Hardware and Software Environment for the users.  The developer environment
is summarized as part of the developer plan.}

\section{Goals}

\subsection{Create a 3D model of a tight space with little to no natural light}

The primary goal of the project is to make accurate models of crevice caves. These often have
no light from the outside and require the caver to bring their own lighting. The space is also
constrained, often narrowing to passages not much larger than what a person can pass through.
The exact format of the output is not yet determined but could be a point cloud or a 3D mesh.

\subsection{Total cost of project is under \$750}

Typically, professional systems for mapping spaces use 3D Li-DAR scanners and cost multiple
thousands of dollars. The total cost of the sensors and hardware used needs to be limited for the
requirements of the capstone, but keeping a low cost would be an important goal regardless of this.
We will be able to reuse equipment purchased by a previous capstone group, which will lower our
expenses, but even if the sensors we already have are included in the budget, the cost should remain
well under \$750. The goal is to lower the barrier to entry for these types of mapping tasks while
maintaining a reasonable level of accuracy.

\subsection{The operator's ability to explore crevice caves is not hindered}

Spelunking is a physically demanding activity. The system should be wearable by the user in a
way that does not impede their ability to navigate through challenging caves. The space that the
system uses should be small and the components be located to minimize disruption of the caver’s
experience. The system should also be light, to avoid fatiguing the user, and the battery charge
should last long enough to avoid having to halt a caving session for recharging.

\subsection{Use of the system does not pose any danger to the operator or others}

There should be no risks to the user caused by the mapping system. Specifically, the system
should not decrease the effectiveness of existing safety equipment such as harnesses or helmets.
Furthermore, the batteries should be stored in a secure manner such that a fall does not cause
further issues related to battery safety . This is related to the previous goal, but the focus of this goal is on the safety of the user and any others who could be affected.

\section{Stretch Goals}

\subsection{Local Data Processing}
Implement the capability of processing the collected data from the sensors on the system in real time.

\subsection{Camera Improvements}
Design a camera mounting system with 30° range of motion, with 1° micro-movement via an interface. Add a RGB camera to capture color information to generate colored 3D models.

\subsection{Phone Application}
Develop an application to access previously made images and 3D mappings.

\subsection{Machine Learning}
Implement a learning mechanism that allows the option to reject images, and will train on accepted images.

\section{Extras}

\wss{For CAS 741: State whether the project is a research project. This
designation, with the approval (or request) of the instructor, can be modified
over the course of the term.}

\wss{For SE Capstone: List your extras.  Potential extras include usability
testing, code walkthroughs, user documentation, formal proof, GenderMag
personas, Design Thinking, etc.  (The full list is on the course outline and in
Lecture 02.) Normally the number of extras will be two.  Approval of the extras
will be part of the discussion with the instructor for approving the project.
The extras, with the approval (or request) of the instructor, can be modified
over the course of the term.}

\newpage{}

\section*{Appendix --- Reflection}

\wss{Not required for CAS 741}

\input{../Reflection.tex}

\begin{enumerate}
    \item What went well while writing this deliverable? 
    \item What pain points did you experience during this deliverable, and how
    did you resolve them?
    \item How did you and your team adjust the scope of your goals to ensure
    they are suitable for a Capstone project (not overly ambitious but also of
    appropriate complexity for a senior design project)?
\end{enumerate}  

\end{document}
